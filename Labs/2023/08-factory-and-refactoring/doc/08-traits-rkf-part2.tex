\documentclass[10pt,aspectratio=169]{beamer}
\usetheme{default}
\setbeamercovered{invisible}
\setbeamertemplate{navigation symbols}{}
\setbeamertemplate{footline}{
    \flushright{\hfill \insertframenumber{}/\inserttotalframenumber}
}

\usepackage{listings}

% User-defined colors.
\definecolor{DarkGreen}{rgb}{0, .5, 0}
\definecolor{DarkBlue}{rgb}{0, 0, .5}
\definecolor{DarkRed}{rgb}{.5, 0, 0}
\definecolor{LightGray}{rgb}{.95, .95, .95}
\definecolor{White}{rgb}{1.0,1.0,1.0}
\definecolor{darkblue}{rgb}{0,0,0.9}
\definecolor{darkred}{rgb}{0.8,0,0}
\definecolor{darkgreen}{rgb}{0.0,0.85,0}

% Settings for listing class.
\lstset{
  language=C++,                        % The default language
  basicstyle=\small\ttfamily,          % The basic style
  backgroundcolor=\color{White},       % Set listing background
  keywordstyle=\color{DarkBlue}\bfseries, % Set keyword style
  commentstyle=\color{DarkGreen}\itshape, % Set comment style
  stringstyle=\color{DarkRed}, % Set string constant style
  extendedchars=true % Allow extended characters
  breaklines=true,
  basewidth={0.5em,0.4em},
  fontadjust=true,
  linewidth=\textwidth,
  breakatwhitespace=true,
  showstringspaces=false,
  lineskip=0ex, %  frame=single
}

\begin{document}
\title{Generic factory and code refactoring:\protect\\A Runge-Kutta-Fehlberg solver\protect\\using traits and concepts\protect\\(part II)}
    \author{Matteo Caldana}
    \date{27/04/2023}

\begin{frame}[plain, noframenumbering]
    \maketitle
\end{frame}

\begin{frame}{Refactoring the RKF solver}
Starting from the solution to the previous lab:
\begin{enumerate}
\item define a class to handle the input options for the \texttt{RKF} class, provide the corresponding setter method and a method to parse options from a GetPot file;
\item implement the Fehlberg12 and the Dormand-Prince methods (\url{https://en.wikipedia.org/wiki/List_of_Runge\%E2\%80\%93Kutta_methods\#Embedded_methods}) and use them to solve the Lorenz system (\url{https://en.wikipedia.org/wiki/Lorenz_system}). Plot the solution both as a function of time and in the phase space;
\item implement a custom factory for the \texttt{ButcherArray} class, so that the actual method can be polymorphically selected at runtime from the GetPot file;
\item replace your custom-defined factory using the generic factory provided in the \texttt{generic-factory} folder, and provide a proxy for registering all RKF methods implemented.
\end{enumerate}
\end{frame}

\end{document}
