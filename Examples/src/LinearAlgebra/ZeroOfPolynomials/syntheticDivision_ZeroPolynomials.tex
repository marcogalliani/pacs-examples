\documentclass[10pt,a4paper]{article}
\usepackage[utf8]{inputenc}
\usepackage[T1]{fontenc}
\usepackage[italian]{babel}
\usepackage{amsmath}
\usepackage{amsthm}
\usepackage{amsfonts}
\usepackage{amssymb}
\usepackage{graphicx}
\usepackage{listings}
\usepackage{color}
\usepackage{hyperref}
\usepackage{a4wide}
\definecolor{mygreen}{RGB}{28,172,0} % color values Red, Green, Blue
\definecolor{mylilas}{RGB}{170,55,241}
\setlength{\parindent}{0pt}
\theoremstyle{definition}% default
\newtheorem{algo}{Algorithm}
\author{Luca Formaggia}
\date{AY 2021-22}
\title{A note on synthetic division and zero of polynomials}
\begin{document}
  \lstset{language=c++,%
     %basicstyle=\color{red},
     breaklines=true,%
     morekeywords={matlab2tikz},
     keywordstyle=\color{blue},%
     morekeywords=[2]{1}, keywordstyle=[2]{\color{black}},
     identifierstyle=\color{black},%
     stringstyle=\color{mylilas},
     commentstyle=\color{mygreen},%
     showstringspaces=false,%without this there will be a symbol in the places where there is a space
     numbers=left,%
     numberstyle={\tiny \color{black}},% size of the numbers
     numbersep=9pt, % this defines how far the numbers are from the text
     emph=[1]{for,end,break},emphstyle=[1]\color{red}, %some words to emphasise
     %emph=[2]{word1,word2}, emphstyle=[2]{style},    
 }
 
 \maketitle
 \section{Synthetic division and Horner algorithm}
 Let
 \[
 p_m(x)=\sum_{i=0}^{m} a_i x^i,
 \]
be a (real or complex) polynomial of degree at most $m$, expressed in terms of the monomial basis.
  The \emph{Synthetic Division} or \emph{Horner's algoritm} reads

\begin{algo}\label{alg:horner}
\emph{Computes $p(z)$}
\begin{lstlisting}
auto horner(a,z)
    b[m-1]=a[m];
    for k=m-2,...0 b[k]+=b[k+1]*z + a[k+1]; 
    return {b[0]*z + a[0],b}; 
\end{lstlisting}
\end{algo}
This version of the algorithm, beside computing $p_m(z)$,
  returns the coefficients $b_j$, $j=0,\ldots m-1$ of the so called \emph{associated polynomial}
\begin{equation}
\label{eq:associated}
q_{m-1}(x;z)=\sum_{j=0}^{m-1}b_j x^j.
\end{equation}
It is a polynomial of degree at most $m$, and the $z$ in $q_{m-1}(x;z)$ indicates
  that the coefficients $b_j$ are the results of the synthetic division applied in $x=z$ (see Algorithm~\ref{alg:horner}). Indeed, the value of the coefficients $b_j$ depends on the chosen point $z$.

We have the following \emph{important result}: the polynomial $q_{m-1}(x;z)$ is the quotient
  of the division of $p_m$ by $x-z$. Which means that
\begin{equation}
\label{eq:quotient}
\boxed{p_m(x)=q_{m-1}(x;z)(x-z) + p_m(z),}
\end{equation}
and, consequently,
\begin{equation}
\label{eq:quotientder}
\boxed{\frac{d}{dx}p_m(x)=\frac{d}{dx}q_{m-1}(x;z)(x-z) + q_{m-1}(x;z),}
\end{equation}
by which,
\begin{equation}
\label{eq:quotientdertwo}
\boxed{\frac{d}{dx}p_m(z)=q_{m-1}(z;z).}
\end{equation}

Therefore, the synthetic division is a tool to compute not only the value at a point $z$, but also the derivative at that point, and the quotient of the division with the factor $x-z$. It provides the basic ingredients of the \textbf{Newton-Horner} algorithm for computing the zeros of a polynomial. But first we need to illustrate the deflation.
\subsection{Deflation}
Let $\{\alpha_j,\ j=1,\ldots,n\}$ be a zeroes of the polynomial $p_n$. Then, $\{\alpha_j,\ j=1,\ldots,n-1\}$ are the zeroes of 
the polynomial
\begin{equation}
p_{n-1}=p_n/(x-\alpha_n).
\end{equation} 
The proof is trivial.
But $p_{n-1}(x)=q_{m-1}(x;\alpha_n)$, according to~\eqref{eq:quotient}, being $p_n(\alpha_n)=0$! This allows to set a very efficient algorithm for 
finding all zeroes of a polynomial.
\subsection{Newton-Horner with deflation}
The algorithm here implemented is Newton-Horner with deflation:

\begin{algo}
    Given the polynomial $p_n$, the initial point $x_0$ and tolerances $tolr$ (tolerance on residual) and $told$
    (tolerance on iteration length)
    do:
    \begin{description}
        \item[] Set $q=p_n$.
        \item[] For $k=1,n$:
        \begin{enumerate}
            \item Perform the Newton iteration: for $n=0,1,\ldots$
            \begin{enumerate}
                \item Compute $q(x_n)$ and $q^\prime(x_n)$ using synthetic division;
                \item Compute $x_{n+1}=x_{n}-q^\prime(x_{n})^{-1}q(x_n)$;
                \item If $|q(x_{n+1})|<tol$ and $|x_{n+1}=x_{n}|$ terminate iteration and set
            $\alpha_k=x_{n+1}$ (the $k$-th zero);
            \end{enumerate}
            \item Set $q=q/(x-x^{(k)})$ (deflation) using the associated polynomial given by the last synthetic division!;
            \item Set $x_0=\overline{\alpha_k}$ (to facilitate finding conjugate zeroes);
        \end{enumerate}
    \end{description}
\end{algo}

This algorithm  exploits the fact that the synthetic division (Horner's algorithm) is able to provide the evaluation at a point of both the polynomial and its derivative and, when the zero is found, also the coefficients of the deflated polynomial! 
\subsection{Some comments}
Newton-Holder algorithm is a very nice and relatively efficient algorithm. However
\begin{itemize}
    \item We have global convergence (i.e. for all $x_0$) only for polynomial with real coefficients under some conditions on the zeroes (look at specialized literature if you want more information). Therefore, in general convergence of the basic algorithm depends on the choice of $x_0$. You may implement techniques like backtracking, or randomization, to better convergence properties. It must be said that in all the tests I have made the basic method worked fine.
    \item The deflation is only approximate! Indeed, we will "divide" not by the real zero, but by the \emph{approximated zero}. This means that we may accumulate errors that may reduce the accuracy of the last computed zeroes: a small value of the \emph{approximated} deflated polynomial may not imply a small value of the original polynomial. Consequently, the algorithm may terminate at a point far from the real zero. 
    However, this may happen only if you compute all zeroes of a polynomial of high degree, and for the last computed zeroes! Normally, you do not have to worry if you have set the tolerances reasonably tight. 
    \item In real polynomials, complex roots come in pairs. Cannot we deflate by $(x-\alpha)(x-\overline{\alpha})$ directly?
    Indeed we can, it is a modification of the basic algorithm that however I have not implemented here.
    \item You have to avoid division by zero in   $q^\prime(x_n)^{-1}p(x_n)$!.
    This is a tricky business. What you can try to do is
    to set a small number at the denominator. For instance, using $p(x_n)/d$ with $d=\epsilon<<1$ if $|q^\prime(x_n)|=0$ and $d=q^\prime(x_n)$ otherwise. 
    \item Operating with complex numbers makes the algorithm more general. But if $x_0$ is a pure real number (i.e. zero imaginary part), then the algorithm will find only the real zeroes. Since we do not know how, in general, how many real zeroes we have, the algorithm may fail to find a zero simply because we do not have other real zeroes! Therefore, we need to modify the algorithm to detect non convergence. Here I have decided to keep things simple, but then you have always to start with an 
    $x_0$ with non-sero imaginary part, unless you know beforehand that \emph{all} zeroes are real (or, at least, the number of real zeroes, since the given algorithm allows to search only for a certain number of zeroes).
    \item When a complex numer is in fact a pure real number? The answer seems straightforward: when the imaginary part is zero. This is true in the ideal world of complex numbers, not in the real world of \emph{floating point complex numbers}. Roundoff errors
    will often cause the imaginary part be very small, but not zero.  A rule of thumb is to consider zero the imaginary part of the
    approximated root found by the algorithm if its absolute value is smaller than the given tolerances. Of course, if you know a priori that the root is real you will consider only the real part of the found approximation.
\end{itemize}


\section{Derivatives}
Leveraging Equations~\eqref{eq:quotientder} and~\eqref{eq:quotient} we can also compute higher derivatives at point $z$. Let's first look to the second derivative
\begin{multline}\label{eq:secderexp}
\frac{d^2p_m}{dx^2}(x)=\frac{d^2q_{m-1}}{dx^2}(x;z)(x-z) + \frac{dq_{m-1}}{dx} (x;z) + \frac{dq_{m-1}}{dx} (x;z)=\\ \frac{d^2}{dx^2}q_{m-1}(x;z)(x-z) + 2 \frac{d}{dx} q_{m-1}(x;z),
\end{multline}
Thus,
\begin{equation}\label{eq:secder}
\frac{d^2p_m}{dx^2}(z)=2 \frac{d}{dx} q_{m-1}(z;z),
\end{equation}

Using~\eqref{eq:quotientder},
\begin{equation}\label{eq:firstderass}
\frac{d}{dx}q_{m-1}(x;z)=\frac{d}{dx}q_{m-2}(x;z)(x-z)+ q_{m-2}(x;z),
\end{equation}
which combined with~\eqref{eq:secder} gives
\begin{equation}\label{eq:secderivative}
\frac{d^2p_m}{dx^2}(z)=2 q_{m-2}(z;z).
\end{equation}
So the second derivative at $z$ is twice the associated polynomial of $q_{m-1}(x;z)$ at the given point. So it can be computed just by using two synthetic divisions.
What about the third derivative? We have, from~\eqref{eq:secderexp}
\begin{multline}\label{eq:secderexptwo}
\frac{d^3p_m}{dx^3}(x)=\frac{d^3}{dx^3}q_{m-1}(x;z)(x-z) +\frac{d^2}{dx^2}q_{m-1}(x;z) + 2 \frac{d^2}{dx^2} q_{m-1}(x;z)=\\
\frac{d^3}{dx^3}q_{m-1}(x;z)(x-z) +3\frac{d^2}{dx^2}q_{m-1}(x;z).
\end{multline}
Using~\eqref{eq:firstderass}
\begin{equation}
\frac{d^2}{dx^2}q_{m-1}(x;z)=\frac{d^2}{dx^2}q_{m-2}(x;z)(x-z)+2\frac{d}{dx}q_{m-2}(x;z),
\end{equation}
and thus
\begin{equation}\label{eq:thirdder}
\frac{d^3p_m}{dx^3}(x)=
\frac{d^3}{dx^3}q_{m-1}(x;z)(x-z) +3\frac{d^2}{dx^2}q_{m-2}(x;z)(x-z)^2 + 6\frac{d}{dx}q_{m-2}(x;z),
\end{equation}
from which, exploiting again~\eqref{eq:quotientder}, we have
\begin{equation}\label{eq:thirdderz}
\frac{d^3}{dx^3}p_m(z)=6 q_{m-3}(z;z).
\end{equation}

In general, for any $n\le m$,
\begin{equation}
\boxed{\frac{d^n}{dx^n}p_m(z)=n! q_{m-n}(z;z).}
\end{equation}

Therefore, all derivatives at a given point $z$ may be computed with a repeated use of synthetic division.


\section{The code in \texttt{polyHolder.hpp}}
The code in \texttt{polyHolder.hpp} contains
\begin{itemize}
    \item The function \lstinline|polyEval()| which uses the basic Holder's algoritm to compute the value of a polynomial at a point;
    \item The class \lstinline|polyHolder|, which is a helper class for the Holder-Newton algorithm and it allows also to compute derivatives of a polynomial (given its coefficients) at a given point.
    \item The function \lstinline|polyRoots()| which computes the roots of a polynomial using the basic \emph{Newton-Holder} procedure.
\end{itemize}

You may argue that I am not respecting the \emph{single responsibility principle} here: the class \texttt{polyHolder} should only provide the synthetic division. The computation of derivatives should be the responsibility of a function (or another class) that uses \texttt{polyHolder}. You are right, but the world is not perfect, and the \emph{principles} should be taken with a grain of salt.


\end{document}